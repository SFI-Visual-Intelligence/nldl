% !TeX spellcheck = en_US
% !TeX program=pdflatex
% !BIB program=biber

%%%%%%%%%%%%%%%%%%%%%%%%%%%%%%%%%%%%%%%%%%%%%%%%%%%%%%%%%%%%%%%%%%%%%%%
% Options for the class:
% 'fullpaper' if you are submitting a 6-page paper or
% 'abstract' if you are submitting a 2-page extended abstract.
% 'review' (is the default) will anonymize the submission and will add the review line numers.
% 'final' will add a mandatory footer on the titlepage which must contain the DOI. This will be provided once the paper is accepted and everything is ready for publication. This footer is only necessary for full papers.
%%%%%%%%%%%%%%%%%%%%%%%%%%%%%%%%%%%%%%%%%%%%%%%%%%%%%%%%%%%%%%%%%%%%%%%
%\documentclass[fullpaper,final]{nldl}
\documentclass[fullpaper]{nldl}
%\documentclass[abstract]{nldl}

% Replace '12345' with the DOI of the publication.
%\DOI{12345}
\paperID{42}
% defaults to current year plus one, or you can set it here
%\confYear{2042}
%%%%%%%%%%%%%%%%%%%%%%%%%%%%%%%%%%%%%%%%%%%%%%%%%%%%%%%%%%%%%%%%%%%%%%%
\geometry{margin=2.5cm}


\usepackage[utf8]{inputenc}
\usepackage{microtype}


%% Math
\usepackage{mathtools}

% Figures
\usepackage{graphicx}


%% Tables
\usepackage{booktabs}


%% Lists
\usepackage{enumitem}


% Algorithms
\usepackage{algorithm}
\usepackage{algorithmic}
\newcommand{\theHalgorithm}{\arabic{algorithm}}


%% Code
\usepackage{listings}
\lstset{
  basicstyle=\small\ttfamily,
  breaklines,
}

% References
\addbibresource{references.bib}


% Links and hyperlinks
\usepackage{hyperref}
\usepackage{url}
\hypersetup{
  colorlinks,
  linkcolor = BrickRed,
  citecolor = NavyBlue,
  urlcolor  = Magenta!80!black,
}



% Replace with your title, authors and affiliation
\title{Northern Lights Deep Learning Conference Template}
\author[1]{John Doe\thanks{Corresponding Author.}}
\author[1,2]{Jane Doe}
\affil[1]{Affiliation 1}
\affil[2]{Affiliation 2}
\affil[ ]{\texttt{\{john.doe, jane.doe\}@affiliation.com}}


\begin{document}
\maketitle

\begin{abstract}
Deep learning is an emerging subfield in machine learning that has in recent years achieved state-of-the-art performance in image classification, object detection, segmentation, time series prediction and speech recognition to name a few.
This conference will gather researchers both on a national and international level to exchange ideas, encourage collaborations and present cutting-edge research.
\end{abstract}

\section{Introduction}
The Northern Lights Deep Learning Conference will take place in Troms{\o}, Norway.
Being 350 km north of the arctic circle, Troms{\o} is the northernmost city in the world with a population above 75,000.
As Troms{\o} is located in the middle of the Northern Lights Oval, the area with highest probability of observing northern lights, there is always a good chance to see them.
In general Troms{\o} has a mild climate for such a northerly destination because of its seaside location and the warming effect of the Gulf stream.
Troms{\o} is easily accessible from other Norwegian cities and from abroad.
More information about Troms{\o} is available at Visit Troms{\o}~\cite{tromso}.

\section{Submission Guidelines}
We invite participants to submit in PDF format either 1) a 2-page extended abstract (non-archival) or 2) a 6-page full paper (included in proceedings) in the area of Deep Learning.
The submissions can be submitted by following the instructions on the conference website.\footnote{\url{https://nldl.org}}
Submissions must be in pdf format (two column, font size 10) using the official NLDL template using the \verb|review| option.
All submissions will undergo double-blind peer review.
It will be up to the authors to ensure the proper anonymization of their paper (see Section~\ref{sec:anon}).
In summary, do not include any names or affiliations, and refer to your own past work in the third-person.
\textbf{Submissions that violate any of these restrictions will be desk-rejected without going further review.}
Accepted contributions will be accepted as contributing talks or poster presentations.

\section{Anonymized Submission}
\label{sec:anon}

To maintain a fair and impartial review process, we require anonymized submissions.
This means that you must carefully follow these guidelines to ensure that your manuscript conceals your identity during the peer review process.

\noindent\textbf{Personal information.}
Remove all author names and any personal identifying information from the manuscript, such as author bios or acknowledgments.
(Note that this does not mean that you cannot cite your previous work.
See below for details.)
Similarly, do not include the names of your affiliations or institutions in the manuscript.
Note that this template \textbf{defaults to the anonymized version} (\verb|review|) that automatically removes authors' names and institutions.

\noindent\textbf{Self-References.}
While it is acceptable to cite your own previous work, do so in the third person to maintain anonymity.
For example, write, ``Smith et al. [1] explored this concept,'' instead of ``In our previous work [1], we explored this concept.''
Do not remove the name of the authors in any of included references.
This information is not considered personal information since it will be considered as any other reference.

\section{Camera Ready}

Use \verb|final| in the options of the class to produce the camera ready version.
This will automatically remove the line numbers and will de-anonymized the submission.

\section{Style Guidelines}

\begin{enumerate}[leftmargin=*]

\item \textbf{Floating elements.}
Floats in a paper allow for flexible placement of figures, tables, and other similar objects.
When including a figure or table, use the \verb|\begin{figure}[tb]| or the \verb|\begin{table}[tb]| environments to automatically create a float.
These environments take an optional location parameter.
You must avoid forcing the floats in place with \verb|!hH|.
Instead, you should let \LaTeX\ determine the optimal position for your figure or table based on the overall layout of the page, aiming to maintain a clean, coherent layout of the text and figures by using \verb|tb| instead.
The locations \verb|t| and \verb|b| suggest to \LaTeX\ that it should place the figure or table at the top or bottom of the page if possible, but will not force it to do so.

Figure captions should be placed at the bottom while table captions at the top to improve the document's readability.
Sample usage is below, with the examples displayed in Fig.~\ref{fig:sample} and Table~\ref{tab:sample}.
  \begin{lstlisting}[gobble=2]
  \begin{figure}[tb]
    \centering
    \includegraphics[width=\linewidth]{example-image-a}
    \caption{This is a sample image.}
    \label{fig:sample}
  \end{figure}

  \begin{table}[tb]
    \centering
    \caption{This is a sample table.}
    \label{tab:sample}
    \begin{tabular}{lcr}
      \toprule
      Column1 & Column2 & Column3 \\
      \midrule
      1 & 2 & 3 \\
      \bottomrule
    \end{tabular}
  \end{table}
\end{lstlisting}

\begin{figure}[tb]
  \centering
  \includegraphics[width=\linewidth]{example-image-a}
  \caption{This is a sample image.}
  \label{fig:sample}
\end{figure}

\begin{table}[tb]
  \centering
  \caption{This is a sample table.}
  \label{tab:sample}
  \begin{tabular}{lcr}
    \toprule
    Column1 & Column2 & Column3 \\
    \midrule
    1 & 2 & 3 \\
    \bottomrule
  \end{tabular}
\end{table}

\item \textbf{Mathematics.}
Mathematical expressions can be inserted into the text either inline or as standalone entities. For inline mathematical expressions, you can use either the \verb|$$...$$| or \verb|\(...\)| delimiters.
However, the \verb|\(...\)| notation is preferred as it offers better error handling.
If there are any syntax errors within the \verb|\(...\)| delimiters, \LaTeX\ provides more explicit error messages that help in quickly spotting and correcting the issue.

For standalone, block equations, use the \verb|\begin{equation}| and \verb|\end{equation}| macros.
This also automatically numbers the equations.
It is recommended that all mathematical formulas or equations be numbered, whether or not they are explicitly referenced in your text.
The equation numbers will assist other researchers who may wish to refer to specific parts of your work in their own publications.


Variables in mathematical expressions should ideally be represented by a single letter, both for subscripts and superscripts.
In cases where more context or clarity is required by including words or multiple letters, you should escape the word using \verb|\mathit|, \verb|\text|, or \verb|\mathrm|.
For example:
\begin{equation}
  \mathbf{w}^{\text{(new)}} = \mathbf{w}^{\text{(old)}} - \alpha \frac{\partial \mathcal{L}(\mathbf{w}^{\text{(old)}})}{\partial \mathbf{w}}.
\end{equation}
But you should prefer to use $\mathbf{w}^{t}$ instead of the word based superscripts.
The mathematics in the manuscript should flow within the language and should not appear as standalone pieces.
You are recommended to review the work of \citet{Knuth1989}.

\item \textbf{Code.}
For inline code references, use the \verb|\texttt|, \verb|\verb|, or \verb|\lstinline| commands.
Larger code blocks can be included using the \verb|verbatim| environment.
If pseudocode or more sophisticated code formatting is needed, use the \verb|algorithm2e| or \verb|listings| packages.

Code blocks that follow the text can appear in between the paragraphs.
For larger pieces that will be referred to in the text, it is better to use an \verb|algorithm| environment and use it as a float element (similar to the figures and tables).

\item \textbf{Citations.}
This style enables textual and parenthetical citations \verb|\citet| and \verb|\citep| through the \verb|natbib=true| option in the \verb|biblatex| package.
This style also provides a possessive or genitive citation macro \verb|\citepos| and \verb|\Citepos| that behaves as the textual citations with the addition of the possessive marker at the end.

Note that citations are not nouns, and you have to use them tied to a noun or a noun-phrase when used as parenthetical citations.
With parenthetical citations, you should use non-breakable spaces (\verb|~|) between the word and the \verb|\cite| macro (see the usage throughout this document for examples).
\Eg, you should write ``We refer to some website~\citep{tromso}.''  instead of ``We refer to~\citep{tromso}.''
Similarly, you can rely on the textual form of the citation instead.

\item \textbf{References.}
This style uses the BibLaTeX package since it offers a modern, flexible, and convenient way to handle bibliographic data.
Ensure to add your bibliographic database (\verb|.bib| file) within the preamble as \verb|\addbibresource{mybibliography.bib}|.
To display the bibliography, use the \verb|\printbibliography| command to print it out.

In this style, we load the \verb|biblatex| package with the \verb|biber| backend.
You must ensure the right chain of compilation: \verb|LaTeX => Biber => LaTeX| (twice).
Many modern \LaTeX\ editors can streamline this process, match your editor preferences accordingly.

\item \textbf{Abbreviations.}
This style provides a set of the most common academic abbreviations that take care of correct punctuation after them.
The macros available are: \verb|\eg|, \verb|\Eg|, \verb|\ie|, \verb|\Ie|, \verb|\cf|, \verb|\Cf|, \verb|\etc|, \verb|\vs|, \verb|\wrt|, \verb|\dof|, \verb|\iid|, \verb|\wolog|, and \verb|\etal|.

\end{enumerate}

\section{Conclusion}
We look forward to see you all in Troms{\o} for an engaging and productive conference!


\printbibliography

\end{document}
